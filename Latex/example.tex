%Most of this information was found in a thorough introduction to latex
% at: http://www.maths.tcd.ie/~dwilkins/LaTeXPrimer/
%This intro gets you into the typing and basics, but is more math focused
%A good all around introduction is available at
% http://www.andy-roberts.net/writing/latex
% formatting instructions
% http://www.andy-roberts.net/writing/latex/formatting
% bibliography stuff
% http://www.andy-roberts.net/writing/latex/bibliographies

\documentclass[a4paper, 10pt]{article}
%start off latex files with the specifications of the entire document,
%[paper, font]{class type}

\usepackage[margin=1in]{geometry} %for 1 inch margins on the edges
\usepackage{setspace} %for double spacing

\title{How-to-LaTeX}
\author{Thomas Lux}
\date{August 9th 2013}

\begin{document}
%use 'begin' to mark the beginning of a part of the paper, for the
%main section you would begin 'document'
\maketitle
%maketitle puts the title information provided into the document

\doublespacing %begin the double spacing


\section{Starters with sections}
``section'' is used to create headers in your document

\subsection{Active characters}
'subsection' starts a single nested section

 The `active characters'  \# \$ \% \& \_ \{ \}
have special purposes within LaTeX. Thus they cannot be produced in
the final document simply by typing them directly. On the rare
occasions when one needs to use the special characters in the final
document, they can be produced by typing the control sequences

  \# \$ \% \& \_ \{ \}

\subsubsection{Buried deep}
You can have up to  double sub-sections, notice that each type of
(sub)section is headed with appropriately sized font.

\subsection*{Hiding the numbers}
If you put an asterisk after the (sub)section, then it hides the
number in the heading.  It can look neater if you don't want the
blatent counting.

\subsection*{Simple LaTeX text commands}

\textup{ This makes the font upright, which is default in latex.}

\textit{ Now the text is in italics.}

\textsl{ This is slanted text, very similar to italics.}

\textsc{ Small caps mode, where everything is upper case with capital
  letters simply taller than lower case.}

\textbf{ Lasly, boldface mode.}


\subsection*{Extra stuff}
When there is a word or phrase that you do not want to be split up,
such as a name, use tilda instead of space to tell LaTeX to keep the
adjacent words together.  This~is~all~together~because~I~used~tilda
\\This statement lands on its own line using the new line character ``\\''\\
Also, when typing out abreviations with periods such as ``Mr.\'' or
``etc.\'' it is a good idea to put a backslash after the period so that
LaTeX doesn't split the abreviation and the following word.

\begin{verbatim}
This is an evironment that only recognizes backslash and end verbatim
as a command.  Everything else is taken verbatim from how you type. No
SpEcIaL ChArAcTeRs apply \ ! @ # $ % ^ & * ( ) _ + = - 
\end{verbatim}


\section*{Using whitespace for asthetic seperation}
 To produce blank space within a paragraph, use hspace or vspace
 followed by the length of the blank space enclosed within braces. The
 length of the skip should be expressed in a unit recognized by
 LaTeX. These recognized units are given in the following table:

 Also you can set the paragraph indentation manually.  It is typically
 set by the document class used, but to override it type

%% \setlength{\parindent}{0} %can't remember what this does, needs update
%% %Probably sets paragraph indentation amount to be 0

\vspace{5 mm}

\begin{enumerate}
\item
    pt  point         (1 in = 72.27 pt)
\item
    pc  pica          (1 pc = 12 pt)
\item
    in  inch          (1 in = 25.4 mm)
\item
    bp  big point     (1 in = 72 bp)
\end{enumerate}

And lists can be built with numbers using enumerate, and without
numbers using itemize.  These are functions that must be called with
begin and end, each item declared with the keyword ``item''.

\begin{itemize}
\item
    cm  centimeter    (1 cm = 10 mm)
\item
    mm  millimeter
\item
    dd  didot point   (1157 dd = 1238 pt)
\item
    cc  cicero        (1 cc = 12 dd)
\item
    sp  scaled point  (65536 sp = 1 pt)
\end{itemize}

\section*{Citing your sources}
Last to be covered, how to cite your sources.  Using a .bib file with
all of your sources saved appropriately (look at example.bib for
format), you will import the bibliography by running LaTeX once. Then
you will run BibTeX on the bibliography now that it has been imported,
then you will run LaTeX twice to reimport the bibliography, and then
lastly properly cite all the sources.  The code for importing the
bibliography is beneath.  If I wanted to cite an example source such
as \cite{examplebib2013}, I would use the cite command with the
article's identifier.  I could also cite multiple articles by comma
seperating the files like \cite{examplebib2013, goossens93}.  That's
about it for LaTeX!

\bibliographystyle{plain} %plain style works for standard BibTeX
                          %formatting and entries (haven't seen othewise)
\bibliography{example} %Cite the name of the bibliography file

\end{document}
%use 'end' to mark the end of a section, you must flag the name of the
%section that is ending, in this case, the entire document

%Lastly create your document by running latex on your .tex file from
%the terminal.  Alternatively, you can run pdflatex to produce a pdf
