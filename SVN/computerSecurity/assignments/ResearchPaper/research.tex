\documentclass[a4paper, 12pt]{article}
%start off latex files with the specifications of the entire document,
%[paper, font]{class type}

\usepackage[margin=1in]{geometry} %for 1 inch margins on the edges
\usepackage{setspace} %for double spacing

\title{Cellphone Data Stealing}
\author{Thomas Lux}
\date{November 25th 2013}

\begin{document} %start doc
\maketitle %insert the title

\doublespacing %begin the double spacing

\section*{Introduction}

 In our modern day network of cell phones and towers, we get to enjoy
seamlessly rolling from one cell tower connection to the next without
ever dropping a call.  As impressive as this on-the-go hand off is,
the need to search for cell towers can actually prompt your phone to
send data to anonymous sources.  Cell phones are programmed to
automatically search for, find, and connect to towers.  A 'ping' that
is  sent as frequently as every 7 seconds \cite{bailey2013your} often
contains data that can be used to compromise personal information of
the cell phone user.  Knowing that cell phones must perform the task
of searching for towers, there is a strong need for a secure way to
about this search.  In this paper we will explore the concept of
making this search process more secure after first understanding how
this came to be a problem.  This includes knowing how towers connect
to cell phones, and the impact users will face if the connection
process is left insecure. If you would like a more in depth history of
how cell phones have developed over the years, you can
see \cite{farley2005mobile}.  

\section*{Connecting to the World}

 Cell phones are an amazing technology.  They have provided us with a
new level of connectivity, and they have grown in popularity immensely
over the past years.  Now, it is to the point where upwards of 90\% of
adults in the United States own at least one cell
phone \cite{bailey2013your}.  Across the world, as many as 3.3 billion 
people use some sort of mobile (cell) phone \cite{wolfeinsecurity}.
As we begin to analyze how massive our cell networks must be, we find
some simple flaws and limitations of this technology that we've nearly
become dependent on. \\  \\  To start, we can look at how the cell
tower network is designed.  One cell tower has approximately a range
of 10 square miles.  Each tower also has around 832 different
frequencies it can use to broadcast and receive.  Each cell phone
takes two signals, one for transmission and one for reception, leaving
us with half the original number of frequencies per tower.  If two
cell phone transmissions are on similar frequencies, the messages
often get distorted.  Given this, two adjacent cell towers are not
able to use frequencies that are exceedingly similar.  Now, after
accounting for all the necessary circumstances we've limited the
number of separate cell phone users to which one tower can connect to
be about 56.  \cite{choatecell}  What we find is that there ends up
being a very dense network of cell towers, and transfers between them
must happen very frequently.


\section*{Devices that can trick your phone}

 Since cell phones spend so much time switching from tower to tower, 
they must constantly search for the next available frequency while on 
the move.  This process of searching for towers in cell phones is 
a point of insecurity.  When mobile phones find a tower, they
immediately send information regarding their unique ID, and a
variety of other private information.  One of the devices that is
capable of 'tricking' a cell phone is called the \textbf{Stingray}.
This device emits a signal that cell phones mistake for a tower and
then receives all of the cell phone's identification
information. \cite{stingrayInfo}  This device is not the only of it's
type.  It and its siblings are frequently utilized by law
enforcement agencies for the tracking of individuals.  Most of the
problems that are introduced by these pieces of equipment are
regarding personal privacy.  As to date, there are not known ways to
manipulate the cell phone information that is received into direct
confidential personal information.  This information does however
provide relatively precise location estimations, allowing the user to
be tracked.  The tracking information is capable of locating users
within a range of 150 feet \cite{bailey2013your}.

\section*{How the tech works}
 Stingray is a box-like device that is also dubbed an ``IMSI
Catcher''.  The device sends out signals that resemble those of cell
towers in order to get cell phones to try and connect to it.  In this
process, the cell phone sends the Stingray device it's International
Mobile Subscriber Number and it's Electronic Serial Number possibly
along with hundreds of other unique digital identifiers \cite{stingrayInfo}.

\section*{How this effects everyone}

 This hasn't become a very large issue in the eye of the public to
this day.  Although many people are concerned with breach of privacy,
this is still not a flaw that many cell providers are intent on
fixing.  The main impact of this tracking technology currently is that
it provides not only our government, but other governments as well,
the ability to track people without their express permission.  This
topic is one that has been approached in more legal
terms \cite{curtiss2011triggering}, but has yet to pass through true
legislation.  The primary legal issues are now switching from the cell
providers to the government.  Mainly, with devices such
as \textbf{Stingray} and \textbf{Triggerfish} \cite{stingrayInfo},
'fake' cell tower signals can be produced, bypassing the need for
private sector cooperation in such acts.

\section*{Impact and Solution}

 This technology has the potential to change the way that cell phones
connect to cell towers.  Over the next few years it is very likely
that legislation will be passed prohibiting law enforcement agencies
from abusing cellular device tracking information.  Also likely in the
coming years is some sort of verification system when a cell phone
connects to a tower.  Once cell service providers realize they can
advertise it as added security, it is likely that a system of
verifications will be added to the tower connection process.  The
chances of this significantly decreasing the performance of the phone
is highly unlikely since the task would only have to be performed on
initial tower linking.  Mobile security in general has been a field
slow to catch on.  Many mobile (aka cellular) devices are still
insecure, and put millions at risk of losing personal
information \cite{wolfeinsecurity}.  Over time, this will change, but
many small flaws like the lack of tower verification still crowd the
platforms.

\bibliographystyle{plain}

\bibliography{research}

\end{document}
