\documentclass{article}

\title{Analyzing Techniques for Robotic Localization Mapping}
\author{Thomas Lux \and Randall Pittman}
\date{Dr. Bouchard, Dr. Shende, Prof. Smith}

\begin{document}

\maketitle

\section{Introduction}

This project will analyze several robotic environment mapping
techniques.  It will attempt to discern the various advantages and
disadvantages to these techniques.  Specifically, it will determine
both the efficiency of speed and the accuracy of the map generated by
these methods.  The methods that will be analyzed include ultrasonic
rangefinding~\cite{montemerlo}, camera image depth triangulation~\cite{thrun}, and the Microsoft
Kinect infrared camera~\cite{gil}.  These methods for mapping will be tested by
developing a simple robot that will autonomously create a
3-dimensional map of its surroundings. We discuss our project in more
details in Section~\ref{project}.



\section{Biographical Sketches}

\begin{description}

\item [Thomas Lux:] Thomas Lux is a rising Sophmore Computer Science
  major interested in attending graduate school to focus on the
  departments of machine learning and robotics.  For many years Thomas
  has been interested in computers and computer science.  Computers
  have provided Thomas a means to understand and create useful tools
  for everyday life.  Thomas aspires to continue advancing his
  knowledge of computers so that in the future he can take part in
  larger scale projects that truly put the mind to the test.

\item [Randall Pittman:] Randall is a Computer Science major with a special interest in
        robotics and mathematics.  Since a young age he has been
        fascinated by various properties in mathematics, and during
        high-school developed a keen interest in computer programming.
        He enjoys being able to write programs that perform various
        tasks that help with mathematics or problem solving.  He has
        an internship over the summer with a software development
        company testing code in Java.  He has developed an interest in
        the problem of robotic environment mapping because of the
        mathematical complexity it involves.   
\end{description}



\section{Project Description}: 

	The goal of this project is to compare several different
        techniques that a robot can use to map its environment.  These
        techniques will be ultrasonic range finding, camera
        triangulation, and the Microsoft Kinect infrared camera.  All
        of these technologies will be implemented through the same
        basic environment mapping algorithm, so that the most
        effective method can be found without data error due to the
        use of different algorithms whose efficiency may vary.  The
        two main portions of data that will be compared are: first,
        the ability of the sensor to accurately map an indoor room or
        building, and second, the time it takes the robot to execute
        the maneuvers required to build this map.  These will be the
        main specifications that will determine the comparative
        capability of these technologies.  Another indirect result of
        these conclusions will be a measure of how cost effective
        these devices are for the purposes of environment mapping.  It
        may help to answer the question of whether it is in fact worth
        the trouble, for example, to purchase an infrared Kinect
        camera instead of an ultrasonic sensor, the latter of which
        may be cheaper and easier to implement.  The question is
        whether the Kinect shows a significantly greater deal of
        accuracy than the ultrasonic sensor.  

	The reasons for the importance of the results of this project
        can be seen particularly in the prevalence of robotic scanning
        and mapping technology in the modern world.  Computer
        scientist Michael Montemerlo begins a paper on a specific
        algorithmic solution to the mapping problem saying, "The
        ability to simultaneously localize a robot and accurately map
        its surroundings is considered by many to be a key
        prerequisite of truly autonomous robots." (1)  In order to
        navigate any given environment, an autonomous robot (a robot
        which is not controlled by a user, but by a computer program)
        must have the ability to understand its surroundings so that
        it can navigate through obstacles.  Obstacle course
        navigation, however, will not be a part of this project; the
        efficient building of an accurate map will be the focus.  Such
        a map is a prerequisite for any given obstacle navigation
        algorithm to complete its task.  Author Arturo Gil also
        describes the importance of efficient map building, "Mobile
        robots must possess a basic skill: the ability to plan and
        follow a path through the environment in an optimal way, while
        avoiding obstacles and computing its location within the map.
        In order to solve this problem, mobile robots require the
        existence of a precise map.  In consequence, map building is
        an important task for autonomous mobile robots." (5210)  Many
        developers of navigation algorithms must test for efficiency
        in an actual environment.  Thus there is demand for simple,
        cost-effective, and accurate solution to the robotic
        implementation aspect of such a project.  What then is the
        best technique to consider putting time and money towards?   

	In a paper discussing some of these techniques, Sebastian
        Thrun mentions some of the sensors that are to most popular to
        use when map building, "To acquire a map, robots must possess
        sensors that enable it to perceive the outside world. Sensors
        commonly brought to bear for this task include cameras, range
        finders using sonar, laser, and infrared technology, radar,
        tactile sensors, compasses, and GPS."  He goes on to say that
        "all these sensors are subject to errors, often referred to as
        measurement noise." (2)  Thus it would seem that the data in
        an analysis of the capabilities of camera sensors, sonar, and
        an infrared sensor is quite relevant to a computer scientist
        implementing a mapping algorithm.  Concerning the type of
        infrared camera, the recent development of Microsoft Kinect
        infrared camera has raised some interest in the robotic
        mapping community due to its depth scanning ability.  Thus
        this project will include testing the implementation of this
        new technology to compare it with some of the more traditional
        methods for map building.  

	The plan for obtaining this data can be divided into two
        segments of work.  First, in order to build a map, a robot is
        needed.  The equipment that will be used to build this robot
        will be the Lego Mindstorms product.  We have already begun
        testing the accuracy of various motors and simple sensors that
        come with the product, though more advanced sensors will be
        used later on.  The motors that come with Lego Mindstorms have
        a relatively high accuracy of spin distance, and this will be
        useful when the robot is instructed to move an exact amount.
        Also, Mindstorms will be used to make a platform capable of
        holding the three sensors mentioned earlier.  The second phase
        of work will involve writing the programs and algorithms that
        can obtain sensor data that is then used to build a map.
        Virtually all code used will be written in the Python
        programming language.  One problem that will need to be
        overcome in this stage is converting the Microsoft Kinect
        depth data to a format that can be used in the Python
        language.  For the camera triangulation, software will be
        needed that is capable of camparing two images from a
        camera(s) in different positions and can then compute the
        distance to various points in the image.  For the ultrasonic
        sensor, one that is more advanced than the one that is
        included in Lego Mindstorms will need to be obtained, thus
        allowing for greater accuracy.  A foreseeable problem may be
        connecting this new sensor to the Mindstorms local computing
        device.  


\section{Goals for the Project}\label{project}

	Efficiency and accuracy are two major factors of robotic
        mapping.  The goals of this project include testing multiple
        means of obtaining environmental data, checking all of the
        data for accuracy, and analyzing this accuracys correspondence
        with the speed in which was mapped.

	Measuring efficiency will involve a set of recordings and a
        set of known data.  For the project one algorithm will be used
        for constructing a virtual map across multiple sensor
        platforms.   The efficiency rating for each style sensor will
        be calculated in a simple manner.  Number of data points
        collected over a certain time interval multiplied by the
        relative spread of the data points.  Relative spread can be
        determined by a ratio of given space to average distance
        between points.  For each type of sensor: ultrasonic range
        finding, camera image depth-triangulation, and Kinect sensor
        reading, the number of points obtained in a pre-specified
        amount of time will be recorded and the corresponding
        efficiency will be calculated.  Once efficiency is calculated,
        accuracy is next.

	Accuracy will be measured with regards to map dimensions.  A
        discrete distribution of each recorded data points distance
        from true map coordinates will be constructed.  Using this
        distribution, the mean, the standard deviation, and the
        relative percent error can be calculated.  Once this data is
        obtained, the accuracy of each sensor can be cross-examined on
        a graph.  Accuracy will be a very heavily weighted factor for
        this project.  All proceeding calculations for localization
        and path planning will be formed from the current working map.

	Once all data on efficiency and accuracy for each type of
        sensor has been obtained, each sensor can be compared on a
        larger scale.  This new perspective will provide relevant
        information for future researchers looking to construct
        autonomous mapping robots.

        Summarizing the above, here is an enumerated list of our
        goals.

\begin{enumerate}

\item      Create and algorithm that builds a map given data points
\item      Localize position based on current point data and current map
\item      Test multiple approaches to environmental mapping
\item      	   Ultrasonic range finding
\item	   RGB camera depth triangulation
\item	   Microsoft Kinect multi-point range finding
\item      Calculate efficiency
\item      	   Efficiency = (data pionts / time) * spread
\item	   Spread = total wall space / (average distance between points)
\item      Log mapping data and create a discrete distribution of recorded points distance from actual points
\item      Graph the data provided by the distribution of each platform
\item      Determine the best platform for overall efficiency and
  accuracy when mapping
\end{enumerate}

\bibliographystyle{plain}
%\bibliographystyle{alpha}
\bibliography{Research}

\end{document}
